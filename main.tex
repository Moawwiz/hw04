%CS-113 S18 HW-3
%Released: 2-March-2018
%Deadline: 16-March-2018 7.00 pm
%Authors: Abdullah Zafar, Waqar Saleem.


\documentclass[addpoints]{exam}

% Header and footer.
\pagestyle{headandfoot}
\runningheadrule
\runningfootrule
\runningheader{CS 113 Discrete Mathematics}{Homework IV}{Spring 2018}
\runningfooter{}{Page \thepage\ of \numpages}{}
\firstpageheader{}{}{}

\boxedpoints
\printanswers
\usepackage[table]{xcolor}
\usepackage{amsfonts,graphicx,amsmath,hyperref,amssymb}
\hypersetup{
    colorlinks=true,
    linkcolor=blue,
    urlcolor=cyan,
}

\title{Habib University\\CS-113 Discrete Mathematics\\Spring 2018\\HW 4}
\author{$<Mo'awwiz Arshad Shaheen>$}  % replace with your ID, e.g. oy02945
\date{Due: 19h, 16th March, 2018}


\begin{document}
\maketitle

\begin{questions}



\question
\begin{parts}
  \part Prove that $1 = -1 \rightarrow 2 = 1$
  
  \begin{solution}
   If premise of an implication is false then it does not matter what value the conclusion holds. The statement will always be true.
   
   Since $1 = -1 $ is False. Then the implication is True. 
  \end{solution}
  
  \part Let $a,b \in \mathbb{N}$. What's wrong with the following proof:
  \begin{align*}
  a &= b\\
  a^2 &= ab\\
  a^2 - b^2 &= ab - b^2\\
  (a-b)(a+b) &= (a-b)b\\
  a+b &= b\\
  a &= 0
  \end{align*}
  \begin{solution}
    If $a=b$ then $(a-b)=0$. In step $5$ both sides are being divided by zero. Division by zero is undefined hence the proof can not be carried further.  
  \end{solution}

  \part What's wrong with the following proof that shows $2^n = 1, n = \{0,1,2,...\}$, using strong induction on $n$:
  
  \textbf{Base step:} When $n=0, 2^0=1$, so the result holds.
  
  \textbf{Inductive step:} Suppose the result holds for $0\leq n \leq k$. We will show that it holds for $n=k+1$, i.e. that $2^{k+1}=1$.
  
  \begin{align*}
      2^{k+1} &= \frac{2^{2k}}{2^{k-1}}\\
      &= \frac{2^k.2^k}{2^{k-1}}\\
      &= \frac{1.1}{1}\\
      &= 1
  \end{align*}

  \begin{solution}
    % Write your solution here
    The following conditions should hold for strong induction:
    \begin{parts}
    \part The base step should hold.
    \part The inductive step shoud hold for for all positive integers $k$ in the conditional statement $[P(0)\wedge P(1)\wedge ... \wedge P(k)]\rightarrow P(k+1)$.
    \end{parts}
    The flaw is present in the inductive step. In the question it is assumed that $k \geq 1$ in order for the exponent in the denominator to be a nonnegative integer which is not possible for $0 \leq n \leq k$. If that were the case the inductive hypothesis could not be applied. 
    
    The base case was checked for $n=0$, so it can not be assumed that $k\geq 1$ when trying to prove for $k+1$ in the inductive step because at $n=1$ the proposition breaks down and does not hold. Hence, the proof is invalid. 
  \end{solution}
\end{parts}

\question Buckminster Fuller once said: ``None of the world’s problems will have a solution until the world’s individuals become thoroughly self-educated."

The world is full of problems, but which ones should you focus on? According to \href{http://8000hours.org}{80000.org}, the most urgent problems are ``not only \textbf{big}, they're also \textbf{neglected} and \textbf{solvable} - the fewer people working on a problem, the easier it is to make a big contribution. An issue can be big but comparatively well-known and crowded, like climate change, or it can be small but neglected, like land use zoning reform, and therefore also worth considering."

We present a list of the biggest, most solvable and most neglected global issues that would benefit the most from your contribution. You are encouraged to read more about how these problems measure up at \href{https://80000hours.org/articles/cause-selection/}{8000hours.org}.
\begin{center}
\begin{tabular}{ c c c}
Biosecurity & Climate change & Promoting effective altruism\\\\
Institutional decision-making & Risk from AI & Nuclear Security\\\\
Developing world health & Land use reform & Factory farming
\end{tabular}
\end{center}

You are given a set of 9 global challenges - to which you may add one more of your choice - and three order relations on them: \textbf{NB}: ``not bigger than", \textbf{NS}: ``not more solvable than" and \textbf{NN}:``not more neglected than". Using all 10 challenges and your intuition, give unique answers for each of the following. (Well-formatted, syntactically sound answers are encouraged) 

\begin{parts}
\part A poset of width 5
\part A decomposition of size 4
\part A lattice
\part A totally-ordered set
\part A well-ordered set
\end{parts}

(You may refer to the \textbf{Appendix} for a list of definitions)
\begin{solution}
    
    let X = $\{a,b,c,d,e,f,g,h,i,j\}$, letting each element $\in X$ represent a challenge respectively. $j$ being the challenge added by the student.\\
    \begin{parts}
    
 
    \part
    $\mathcal{P} = (X, NB)\\ NB=\{(a,a),(b,b),(c,c),(d,d),(e,e),(f,f),(g,g),(h,h),(i,i),(j,j)\\,(a,b),(c,d),(e,f),(g,h),(i,j)\}$ \\
    \part
    $\mathcal{P} = (X, NB) \\\ NB = \{(a,a),(b,b),(c,c), (a,b),(b,c),(a,c),
    (d,d),(e,e),(d,e)
    (f,f),(g,g),(f,g)
    (h,h),(i,i)\\,(j,j),(h,i),(i,j),(h,j)\}$\\
    $C_1 = \{a,b,c\}, C_2 = \{d,e\}, C_3 = \{f,g\}, C_4 = \{h,i,j\}$ \\
    \part
     $\mathcal{P} = (X, NB)\\ NB=\{(a,a),(b,b),(c,c),(d,d),(e,e),(f,f),(g,g),(h,h),(i,i),(j,j),\\ 
     (j,a),(j,b),(j,c),(j,d),(j,e),(j,f),(j,g),(j,h),(j,i) \\
     (b,a),(c,a),(d,a),(e,a),(f,a),(g,a),(h,a),(i,a)\\ 
     (i,h),(h,g),(g,f),(f,e),(e,d),(d,c),(c,b)\} $ \\
     \newpage
    \part
   $ \mathcal{P} = (X, NB)$
    \[
  NB=
  \left[ {\begin{array}{cccccccccc}
   1 & 1 & 1 & 1 & 1 & 1 & 1 & 1 & 1 & 1\\
   0 & 1 & 1 & 1 & 1 & 1 & 1 & 1 & 1 & 1\\
   0 & 0 & 1 & 1 & 1 & 1 & 1 & 1 & 1 & 1\\
   0 & 0 & 0 & 1 & 1 & 1 & 1 & 1 & 1 & 1\\
   0 & 0 & 0 & 0 & 1 & 1 & 1 & 1 & 1 & 1\\
   0 & 0 & 0 & 0 & 0 & 1 & 1 & 1 & 1 & 1\\
   0 & 0 & 0 & 0 & 0 & 0 & 1 & 1 & 1 & 1\\
   0 & 0 & 0 & 0 & 0 & 0 & 0 & 1 & 1 & 1\\
   0 & 0 & 0 & 0 & 0 & 0 & 0 & 0 & 1 & 1\\
   0 & 0 & 0 & 0 & 0 & 0 & 0 & 0 & 0 & 1\\
  \end{array} } \right]
\]
    \part
    $\mathcal{P} = (X, NS)$
    \[
  NS=
  \left[ {\begin{array}{cccccccccc}
   1 & 1 & 1 & 1 & 1 & 1 & 1 & 1 & 1 & 1\\
   0 & 1 & 1 & 1 & 1 & 1 & 1 & 1 & 1 & 1\\
   0 & 0 & 1 & 1 & 1 & 1 & 1 & 1 & 1 & 1\\
   0 & 0 & 0 & 1 & 1 & 1 & 1 & 1 & 1 & 1\\
   0 & 0 & 0 & 0 & 1 & 1 & 1 & 1 & 1 & 1\\
   0 & 0 & 0 & 0 & 0 & 1 & 1 & 1 & 1 & 1\\
   0 & 0 & 0 & 0 & 0 & 0 & 1 & 1 & 1 & 1\\
   0 & 0 & 0 & 0 & 0 & 0 & 0 & 1 & 1 & 1\\
   0 & 0 & 0 & 0 & 0 & 0 & 0 & 0 & 1 & 1\\
   0 & 0 & 0 & 0 & 0 & 0 & 0 & 0 & 0 & 1\\
  \end{array} } \right]
\]
    \end{parts}
    
  \end{solution}


  
\question 
A poset $(R, \preccurlyeq)$ is \textbf{well-founded} if there is no infinite decreasing sequence of elements in the poset, i.e. elements $x_1, x_2, \cdots, x_n$ such that $\cdots \prec x_n \prec \cdots  \prec x_2 \prec x_1$. A poset $(R, \preccurlyeq)$ is \textbf{dense} if for all $x \in S$ and $y \in S$ with $x \prec y$, there is an element $z \in R$ such that $x \prec z \prec y$.

Show that the set of strings of lowercase English letters with lexicographic order is neither well-founded nor dense.


  \begin{solution}
    Supposing the chain of string of lowercase English letters to be:
    \begin{center}
     $... aaaaab \preccurlyeq aaab \preccurlyeq aab \preccurlyeq ab$
    \end{center}
 
    This serves as a counter example proving that it is not well-founded since it is infinite and decreasing.\newline
    Furthermore, it is also is not dense  i.e. there is no string between $a$ and $aa$.
  \end{solution}

\question
     Let $S$ and $T$ be two partial orders on a set $A$. Define a new relation $R$ on $A$ by $(x,y)\in R$ iff both $(x,y) \in S$ and $(x,y) \in T$. Prove that $R$ is a partial order on $A$.

    
     \begin{solution}
     For $R$ to be a partial order on set $A$ it has to be reflexive, anti-symmetric, and transitive.
     
     In our case, $S \cap T = R$
     \begin{parts}
     \part \textbf{Reflexive:}
     
     Every pair of the form $(a,a)$ will be present in $S$ and $T$ both because they are reflexive.
     
     And for every pair of the form $(a,a)\in S$ and $(a,a)\in T$ the pair will be present in $R$.
     
     Since all pairs of the form $(a,a) \in R$ hence $R$ is reflexive.
     
     \part \textbf{Symmetric:}
     
     For all pairs of the form $(a,b)$ present in $S$ and $T$, all corresponding pairs of form $(b,a)$ will be not present in $S$ and $T$ because they are antisymmetric.
     
     And for all pairs of the form $(a,b)$ present in $S$ and $T$, and their corresponding pairs of form $(b,a)$ will not be present in $R$. 
     
     Since all pairs of form $(a,b)$ contained in $R$ will not have a corresponding pair of the form $(b,a)$, $R$ will be antisymmetric. 
     
     \part \textbf{Transitive:}
     
     All pairs of the form $(a,b)$ and $(b,c)$ in $S$ and $T$ means a pair of form $(a,c)$ will also exist in both.
     
     Since $S \cap T = R$ therefore, for every all pairs of form  $(a,b)$ and $(b,c)$ in $S$ and $T$ will be present in $R$. And for each of the pairs a pair of form $(a,c)$ will also be present making $R$ transitive.
     
      If we don't have the pairs of form $(a, b)$ and $(b,c)$ in any one of relation $S$ or $T$ or $R$, the antisymmetry will be proved automatically.
     
     \end{parts}
     \end{solution}

\question Your overly-attached girlfriend / boyfriend has concocted a cruel gae to keep you around forever. The game involves piles of excuses - each pile twice as high as the last - of all the times you didn't hang out with them. Stacked in increasing order, the piles look truly endless! 

\begin{figure}[ht]
  \centering
  \includegraphics{excuses.png}
  \caption{That escalated quickly.}
  \label{fig:Piles of excuses}
\end{figure}

The game of \textbf{Gazillion Excuses} is a turn based 2-player game such that at each turn, a player removes a non-zero number of excuses from one of the piles. The player who removes the last excuse wins. Your partner thinks that a sufficiently large number of piles (think gazillion) would keep you playing forever. Little do they suspect that a Discrete Mathematician like yourself has formidable proof skills! Given that you are allowed to go first, \textbf{prove using induction} that you will always win the game of Gazillion Excuses. 


\begin{solution}

For 
$a \oplus b$
  \end{solution}



\newpage
\appendix

\section{Appendix}
Let $\mathcal{P}= (P, \preccurlyeq)$,

\textbf{Chain:} A chain, $C\subseteq P$, is a subset of mutually comparable elements of $P$.\\\\
\textbf{Anti-Chain:} An anti-chain, $A \subseteq P$, is a subset of mutually incomparable elements of $P$.\\\\
\textbf{Width:} The maximum cardinality of an anti-chain of $\mathcal{P}$.\\\\
\textbf{Decomposition:} A decomposition $C$ of $\mathcal{P}$ into chains, is a family $C = \{C_1,C_2,...,C_q\}$ of disjoint chains such that their union is $P$. The size of a decomposition is the number of chains in it.










\end{questions}

\end{document}
